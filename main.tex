\documentclass[master=ucll,11pt,dutch,twoside]{kulemt}
%\documentclass[master=ucll,11pt,dutch,twoside]{ucllfilip}

%master is hetgene dat gedefinieerd staat in de .cfg file
%masteroption is hetgene bij ELT gedeelte staat als optie
%and command is voor een 2de lijn bij te voegen
%%  author={Filip Vanden Eynde \and R0363898 \and 3TX/1},
\setup{
	title={Beste masterproef ooit al geschreven},
  author={Filip Vanden Eynde \and R0363898 \and 3TX/1},
  promotor={Filip Promotor},
 assessor={Filip Assessor},
  company={Otago Polytechnic},
  location={Dunedin, Nieuw-Zeeland},
  textsubject={Eindwerk voorgedragen tot het behalen van mijn eindcijfer voor het vak Beleggen anders bekeken voor het diploma},
  assistant={Filip Assistent}
  }
  
\newcommand{\stagebegeleider}{Griet Barrezeele}
\newcommand{\stagementoreen}{Torleif West}
\newcommand{\stagementortwee}{Siau-juin Lim}
\newcommand{\stagebedrijf}{Otago Polytechnic}
\newcommand{\stagelocatie}{Dunedin, Nieuw-Zeeland}
\newcommand{\exactadresofinternshiplocation}{Forth Street, Private Bag 1910, Dunedin, 9054, New Zealand}
\newcommand{\quotes}[1]{``#1''}
  
% De volgende \setup mag verwijderd worden als geen fiche gewenst is.
%\setup{filingcard,
 % translatedtitle={The best master's thesis ever},
 % udc=621.3,
  %keywords={Stageverslag, Dunedin, Nieuw-Zeeland, UCLL, schooljaar 2016-2017, Filip Vanden Eynde},
  %shortabstract={Hier komt een heel bondig abstract van hooguit 500
   % woorden. \LaTeX\ commando's mogen hier gebruikt worden. Blanco lijnen
   % (of het commando \texttt{\string\pa r}) zijn wel niet toegelaten!
   % \endgraf \lipsum[2]}}
	
% Verwijder de "%" op de volgende lijn als je de kaft wil afdrukken
%\setup{coverpageonly}
%\write18{pdflatex masterproefcoverpage -aux-directory=outputfoldercoverpage}
%\write18{pdflatex masterproefcoverpage}

\newcommand{\kleurlinkstableofcontents}{blue}

% Kies de fonts voor de gewone tekst, bv. Latin Modern
%\setup{font=palatino}

\setup{font=utopia}

\setup{inputenc=utf8}

\usepackage{ifthen}
\newboolean{pdfoutput}
\setboolean{pdfoutput}{true}
%\setboolean{pdfoutput}{false}

% Hier kun je dan nog andere pakketten laden of eigen definities voorzien

\usepackage{kulemtx}
\chapterstyle{kulemtman}
\kulemtmanToC{}

\usepackage{booktabs}
\usepackage{colortbl}
\usepackage[table,dvipsnames]{xcolor}
\usepackage{subcaption}

%\usepackage{geometry}
%\usepackage{pdflscape}



%\begin{comment}

%for bibliography
\usepackage[backend=biber,
%%style=alphabetic,
style=numeric,
%%style=ieee,
%%style=ieee-alphabetic,
citestyle=numeric,
%%refsection=chapter,
%refsegment=chapter,
%defernumbers=true,
%%sorting=ynt,
sorting=nty,
sortcites, % Dit zorgt ervoor dat als je meerdere referenties toont zoals [9,1] dat dit wordt getoond als [1,9] dus numeriek gesorteerd
]{biblatex}
\addbibresource{referentiesstage.bib}

%\defbibheading{subbibliography}{%
%	\section*{References for Chapter \ref{refsection:
%			,→ \therefsection}}}


%\end{comment}



% Dit stylet de bibliografie heading
\begin{comment}


\defbibheading{bibliography}[\bibname]{%
	\section{#1}%
	\markboth{#1}{#1}}
\end{comment}


%indexes
%voor indexen te maken
\usepackage{imakeidx}
\makeindex[
%columns=3, % geeft layout van 3 kolommen
title=Alphabetical Index, % verandert de default title = index
intoc] % voegt de indexpagina aan de inhoudsopgave toe

\usepackage{import}
\usepackage{enumitem}
\usepackage{pdfpages}
%\usepackage{todo}
\usepackage{todonotes}

%provides the multicols environment which typesets text into multiple columns.
\usepackage{multicol}
\usepackage[official]{eurosym}

%-------------------------------------------------------------------------------------------------------------------------
% pas plaats voor en na headings aan gebruikmakend van package titlesec
%-------------------------------------------------------------------------------------------------------------------------

\usepackage{titlesec}
%\titlespacing\section{0pt}{12pt plus 4pt minus 2pt}{0pt plus 2pt minus 2pt}
%\titlespacing\subsection{0pt}{12pt plus 4pt minus 2pt}{0pt plus 2pt minus 2pt}

%------------------------------------------------------------------------

% Define a set of colours for syntax highlighting

\definecolor{MyColor1}{rgb}{0.2,0.4,0.6} %mix personal color
\definecolor{azure(colorwheel)}{rgb}{0.0, 0.5, 1.0}
\newcommand{\textb}{\color{Black} \usefont{OT1}{lmss}{m}{n}}
\newcommand{\blue}{\color{MyColor1} \usefont{OT1}{lmss}{m}{n}}
\newcommand{\blueb}{\color{MyColor1} \usefont{OT1}{lmss}{b}{n}}
\definecolor{purpleexcel}{rgb}{0.4392156862745,0.18823529,0.62745098039}
\definecolor{dkgreen}{rgb}{0,.6,0}%
\definecolor{dkblue}{rgb}{0,0,.6}%
\definecolor{dkyellow}{cmyk}{0,0,.8,.3}%
\definecolor{lightgray}{rgb}{0.95, 0.95, 0.95}%
\definecolor{darkgray}{rgb}{0.4, 0.4, 0.4}%
\definecolor{editorGray}{rgb}{0.95, 0.95, 0.95}%
\definecolor{editorOcher}{rgb}{1, 0.5, 0}%
\definecolor{editorGreen}{rgb}{0, 0.5, 0}%
\definecolor{orange}{rgb}{1,0.45,0.13}%
\definecolor{olive}{rgb}{0.17,0.59,0.20}%
\definecolor{brown}{rgb}{0.69,0.31,0.31}%
\definecolor{purple}{rgb}{0.38,0.18,0.81}%
\definecolor{lightblue}{rgb}{0.1,0.57,0.7}%
\definecolor{lightred}{rgb}{1,0.4,0.5}%
\definecolor{ChapBlue}{rgb}{0.00,0.65,0.65}

%hier komen alle commando's en constante variabelen voor in de opdracht te gebruiken.

\graphicspath{{images/}} % Set the default folder for images
%I usually write something like this, so the entry is hyperlinked for onscreen reading but there's also a footnote to the URL for paper output.
\newcommand\fnurl[2]{%
	\href{#2}{#1}\footnote{\url{#2}}%
}
%\setlist[itemize,1]{leftmargin=*,topsep=0pt}
%\setlist[enumerate,1]{leftmargin=*,topsep=0pt}
\setlist{nosep}
%\newcommand{\tabelgeometry}{%
%\newgeometry{total={7.7in, 9.5in},top=0.7in,bottom=0.7in}
%}

\renewcommand\mempostaddapppagetotochook{\cftinserthook{toc}{BREAK}}
\cftinsertcode{BREAK}{\changetocdepth{-10}}
\let\normalchangetocdepth\changetocdepth % needed for later

\makeatletter
\newcommand\appendixtableofcontents{
	\begingroup
	\let\changetocdepth\@gobble
	\normalchangetocdepth{-10}
	\cftinsertcode{BREAK}{\normalchangetocdepth{3}}
	\renewcommand\contentsname{Appendices overzicht}
	\tableofcontents*
	\endgroup
}
\makeatother


%\chapterstyle{BlueBox}

\ifthenelse {\boolean{pdfoutput}}
	{
	\usepackage[
	%hidelinks = true,
	%urlcolor=magenta,
    pdfusetitle,colorlinks,urlcolor=blue,linkcolor=\kleurlinkstableofcontents,
	plainpages=false,bookmarks=true,bookmarksnumbered,linktoc=all,pdffitwindow=true,
	pdftex,pdfauthor={\theauthor},pdftitle={\thetitle},pdfsubject={\thesubject},
	pdfkeywords={\thekeywords},pdfproducer={\theauthor},pdfcreator={\theauthor}
	]{hyperref}	
	}{

\usepackage[
    pdfusetitle,
    colorlinks,
    %urlcolor=blue,
    urlcolor=black,
    %linkcolor=\kleurlinkstableofcontents,
    linkcolor=black,
    citecolor=black,
	plainpages=false,
	bookmarks=true,
	bookmarksnumbered,
	linktoc=all,
	pdffitwindow=true,
	pdftex,
	pdfauthor={\theauthor},
	pdftitle={\thetitle},
	pdfsubject={\thesubject},
	pdfkeywords={\thekeywords},
	pdfproducer={\theauthor},
	pdfcreator={\theauthor}
	]{hyperref}	

}

%for plural forms of acronyms
\usepackage{acronym}

\usepackage{xparse}
\DeclareDocumentCommand{\newdualentry}{ O{} O{} m m m m } {
	\newglossaryentry{gls-#3}{name={#5},text={#5\glsadd{#3}},
		description={#6},#1
	}
	\makeglossaries
	\newacronym[see={[Glossary:]{gls-#3}},#2]{#3}{#4}{#5\glsadd{gls-#3}}
}

\usepackage[acronym, toc]{glossaries}
\usepackage{glossary-mcols}
\makeglossaries
\import{./}{glossary.tex}
\import{./}{acronyms.tex}
%\import{glossary}


%%%%%%%
% Om wat tekst te genereren wordt hier het lipsum pakket gebruikt.
% Bij een echte masterproef heb je dit natuurlijk nooit nodig!
\IfFileExists{lipsum.sty}%
 {\usepackage{lipsum}\setlipsumdefault{11-13}}%
 {\newcommand{\lipsum}[1][11-13]{\par Hier komt wat tekst: lipsum ##1.\par}}
%%%%%%%

%\includeonly{hfdst-n}
\begin{document}

\begin{preface}
  Dit is mijn dankwoord om iedereen te danken die mij bezig gehouden heeft.
  Hierbij dank ik mijn promotor, mijn begeleider en de voltallige jury.
  Ook mijn familie heeft mij erg gesteund natuurlijk.
  %\quotes{Otago Polytechnic}
  %stagebegeleider \stagebegeleider
\end{preface}

\tableofcontents*

\begin{abstract}
  In dit \texttt{abstract} environment wordt een al dan niet uitgebreide
  samenvatting van het werk gegeven. De bedoeling is wel dat dit tot
  1~bladzijde beperkt blijft.

  \lipsum[1]
    \todo[inline]{Samenvatting schrijven}

\end{abstract}

% Een lijst van figuren en tabellen is optioneel
%\listoffigures
%\listoftables
% Bij een beperkt aantal figuren en tabellen gebruik je liever het volgende:
\listoffiguresandtables
%\todo[inline]{Nakijken of alle labels van figuren en tabellen uniek en duidelijk gedefinieerd zijn?}

% De lijst van symbolen is eveneens optioneel.
% Deze lijst moet wel manueel aangemaakt worden, bv. als volgt:
\chapter{Lijst van afkortingen en symbolen}
\section*{Afkortingen}
\begin{flushleft}
  \renewcommand{\arraystretch}{1.1}
  \begin{tabularx}{\textwidth}{@{}p{12mm}X@{}}
    LoG   & Laplacian-of-Gaussian \\
    MSE   & Mean Square error \\
    PSNR  & Peak Signal-to-Noise ratio \\
  \end{tabularx}
\end{flushleft}
\section*{Symbolen}
\begin{flushleft}
  \renewcommand{\arraystretch}{1.1}
  \begin{tabularx}{\textwidth}{@{}p{12mm}X@{}}
    42    & ``The Answer to the Ultimate Question of Life, the Universe,
            and Everything'' volgens de \cite{h2g2} \\
    $c$   & Lichtsnelheid \\
    $E$   & Energie \\
    $m$   & Massa \\
    $\pi$ & Het getal pi \\
  \end{tabularx}
\end{flushleft}

%\todo[inline]{Op het einde de list of todos verwijderen.}
%\listoftodos

% Nu begint de eigenlijke tekst
\mainmatter

%allemaal new stuff
\clearforchapter

%end allemaal new stuff

%----------------------------------------

%origineel
%\chapter{Inleiding}
\label{inleiding}
%\thispagestyle{chapternohead}
%\pagestyle{ruledfilip}
In dit hoofdstuk wordt het werk ingeleid. Het doel wordt gedefinieerd en er
wordt uitgelegd wat de te volgen weg is (beter bekend als de rode draad).

Als je niet goed weet wat een masterproef is, kan je altijd
Wikipedia\cite{wiki} eens nakijken.

\section{Lorem ipsum 4--5}
\lipsum[4-5]

\section{Lorem ipsum 6--7}
\lipsum[6-7]



%%% Local Variables: 
%%% mode: latex
%%% TeX-master: "masterproef"
%%% End: 


%\chapter{Inleiding}
\label{inleiding}
%\thispagestyle{chapternohead}
%\pagestyle{ruledfilip}
In dit hoofdstuk wordt het werk ingeleid. Het doel wordt gedefinieerd en er
wordt uitgelegd wat de te volgen weg is (beter bekend als de rode draad).

Als je niet goed weet wat een masterproef is, kan je altijd
Wikipedia\cite{wiki} eens nakijken.

\section{Lorem ipsum 4--5}
\lipsum[4-5]

\section{Lorem ipsum 6--7}
\lipsum[6-7]



%%% Local Variables: 
%%% mode: latex
%%% TeX-master: "masterproef"
%%% End: 

%\import{texfiles/include-chapter1/}{chapter1.tex}

\include{hfdst-1}
\include{hfdst-2}
% ... en zo verder tot
\chapter{Het laatste hoofdstuk}
\label{hoofdstuk:n}
Een hoofdstuk behandelt een samenhangend geheel dat min of meer op zichzelf
staat. Het is dan ook logisch dat het begint met een inleiding, namelijk
het gedeelte van de tekst dat je nu aan het lezen bent.

\section{Eerste onderwerp in dit hoofdstuk}
De inleidende informatie van dit onderwerp.

\subsection{Een item}
De bijbehorende tekst. Denk eraan om de paragrafen lang genoeg te maken en
de zinnen niet te lang.

Een paragraaf omvat een gedachtengang en bevat dus steeds een paar zinnen.
Een paragraaf die maar \'e\'en lijn lang is, is dus uit den boze.

\section{Tweede onderwerp in dit hoofdstuk}
Er zijn in een hoofdstuk verschillende onderwerpen. We zullen nu
veronderstellen dat dit het laatste onderwerp is.

\section{Besluit van dit hoofdstuk}
Als je in dit hoofdstuk tot belangrijke resultaten of besluiten gekomen
bent, dan is het ook logisch om het hoofdstuk af te ronden met een
overzicht ervan. Voor hoofdstukken zoals de inleiding en het
literatuuroverzicht is dit niet strikt nodig.



%%% Local Variables: 
%%% mode: latex
%%% TeX-master: "masterproef"
%%% End: 

\include{besluit}

% Indien er bijlagen zijn:
\appendixpage*          % indien gewenst
\appendix

%allemaal new stuff

%\appendixpage          % indien gewenst
%\appendixtableofcontents
%end allemaal new stuff

%nieuw
%\import{bijlagen/}{bijlage1_opdrachtenlijst.tex}

\include{app-A}
% ... en zo verder tot
\include{app-n}

%\includepdf[pages=-]{bijlagen/website_audit_filip_eline_belgian_interns.pdf}


\backmatter
% Na de bijlagen plaatst men nog de bibliografie.
% Je kan de  standaard "abbrv" bibliografiestijl vervangen door een andere.
%\bibliographystyle{abbrv}
%\bibliography{referenties}

\pagestyle{afterbackmatterf}

%\chapter{Bibliografie}


%for bibLatex
%\printbibliography[heading=bibintoc,title={Whole bibliography}] %Prints the entire bibliography with the titel "Whole bibliography"

%hiermee worden alle entries getoond, zelfs als ze niet vermeld worden in de tekst.
%\nocite{*}

%Filters bibliography
%\printbibliography
\printbibliography[title={Referenties}]
\pagestyle{afterbackmatterf}


%\printbibliography[heading=subbibintoc,type=online,title={Sites only}]
%\todo[inline]{Gebruikte bronnen en sites nog als bibliografie toevoegen achteraan.}

%\printbibliography[heading=subbibintoc,type=article,title={Articles only}]
%\printbibliography[heading=subbibintoc,type=book,title={Books only}]

%op aparte pagina zo schrijven dan
%\printbibliography[type=book,title={Books only}]
%\printbibliography[keyword={physics},title={Physics-related only}]
%\printbibliography[keyword={latex},title={\LaTeX-related only}]

\chapter{Glossaries}

\import{./}{importglossariesvolledig.tex}

\chapter{Index}

\printindex

\end{document}

%%% Local Variables: 
%%% mode: latex
%%% TeX-master: t
%%% End: 
